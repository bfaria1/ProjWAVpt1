\section{Metodologia}

A princípio, foram coletados, previamente, os dados provenientes dos cientistas cidadãos através do portal Wikiaves (WikiAves, 2020), além disso, também aqueles provenientes de cientistas pelas coleções: Carnegie Museum Birds Collection (CM-Birds), Coleção de Sons (CSUEL), Fonoteca Neotropical Jacques Vielliard (FNJV), Museum of Comparative Zoology (HU-Zoo), Coleção Zoológica de Referência da Seção de Vírus Transmitidos por Artrópodos - Banco de Aves (IAL-aves), Coleção de Aves (MCP-Aves), Museum of Vertebrate Zoology - Brazilian records (MVZ\_BR), Coleção de Aves do Museu de Zoologia (MZUEL-Aves), OBIS Brasil (OBIS\_BR), Sistema de Informação do Programa Biota/Fapesp (SinBiota), NMNH Extant Specimen and Observation Records (US-Animalia), Coleção de Aves do Museu de Zoologia da UNICAMP (ZUEC-AVE), Coleção Audiovisual do Museu de Zoologia "Adão José Cardoso" da UNICAMP - Coleção de vídeos (ZUEC-VID) disponíveis na rede speciesLink (http://www.splink.org.br). Com propósitos de equiparação dos bancos de dados, todas as espécies foram identificadas de acordo com o a Lista comentada das aves do Brasil pelo Comitê Brasileiro de Registros Ornitológicos (PIACENTINI, 2015). As informações foram coletadas no período de janeiro a fevereiro de 2020.

Para responder às questões supracitadas, serão utilizadas as seguintes variáveis oriundas dos dois bancos de dados: o número de registros de cada espécie por município, o número de espécies registradas por município e suas versões log-transformadas (base 10). O número de habitantes, a área de cada município e as coordenadas geográficas de suas sedes serão obtidas no sítio do IBGE (2020). Os biomas predominantes em cada município serão determinados de acordo com o sítio MapBiomas (2020).

A partir desses dados, análises univariadas descreverão a distribuição de cada descritor para os municípios valendo-se de estatísticas de posição (média, mediana), dispersão (amplitude, interquartis, desvio-padrão), que transmitirão a variabilidade dos dados, e distribuições de frequências, que será visualizada por meio de histogramas (GOTTELI & ELLISON et al., 2016, BORCARD et al., 2018). Será utilizada a ferramenta Microsoft Excel® (2019) para realizar as análises descritas acima.
Análises bivariadas descreverão as relações das variáveis pareadas, por meio de correlação e regressão, em versões paramétricas ou não (GOTTELI & ELLISON et al., 2016, BORCARD et al., 2018). Comparações entre curvas serão feitas por análise de covariância (GOTTELI & ELLISON et al., 2016, BORCARD et al. 2018). Elas serão realizadas por meio dos programas Microsoft Excel® e Past (HAMMER, Ø, 2019).

Análises multivariadas descreverão as relações de similaridade entre municípios quanto à constituição específica e quanto aos valores dos fatores explanatórios (GOTTELI & ELLISON et al., 2016, BORCARD et al., 2018). Para esta dimensão de analise utilizar-se-á, por meio do programa Past, o índice de Jaccard (BORCARD et al., 2018), que abrange a dissimilaridade entre duas amostras baseando-se na presença de espécies em certa região, a partir das matrizes fornecidas por tal será possível compara-las por meio do teste de Mantel (BORCARD et al., 2018). Este método de análise, também, permitirá que sejam comparadas a distância euclidiana por município, que será obtida a partir dos bancos de dados, com a distância geográfica já conhecida entre eles. Mapas temáticos apresentarão as variáveis em sua distribuição geográficas. As abordagens discriminadas serão conduzidas com o programa R (R CORE TEAM, 2013).


\subsection{Materiais e Métodos}

\subsection{Etapas da pesquisa}