\documentclass[12pt]{extarticle}
\usepackage[brazil]{babel}
\usepackage{graphicx, hyperref}
\usepackage{framed}
\usepackage[normalem]{ulem}
\usepackage{amsmath}
\usepackage{amsthm}
\usepackage{amssymb}
\usepackage{amsfonts}
\usepackage{enumerate}
\usepackage[utf8]{inputenc}
\usepackage[top=1 in,bottom=1in, left=1 in, right=1 in]{geometry}
\usepackage{graphicx}
\usepackage[skip=2pt]{caption}
\usepackage{blindtext}
\usepackage[fixlanguage]{babelbib}
\usepackage[alf]{abntex2cite}
\bibliographystyle{abntex2-alf}
\captionsetup{font={small}} 
\graphicspath{{Figuras/}}
\usepackage{appendix}
\newcommand{\asp}[1]{``#1"}
\usepackage[colorinlistoftodos]{todonotes}
\newcommand{\nota}[1]{\todo[color=black!10, bordercolor=white!100, linecolor = black!20, size =\scriptsize]{#1} }
\definecolor{mypink1}{rgb}{0.858, 0.188, 0.478}
\definecolor{mypink2}{RGB}{219, 48, 122}
\definecolor{mypink3}{cmyk}{0, 0.7808, 0.4429, 0.1412}
\definecolor{mygray}{gray}{0.6}
\newcommand{\arb}{um elemento arbitrário }
\newenvironment{resposta}{ \color{mygray}}{}
\newcommand{\true}{\textcolor{red}{\textbf{\textit{V}}}}
\newcommand{\false}{\textcolor{red}{\textbf{\textit{F}}}}
\newcommand{\keys}[1]{\{#1\}}
\newenvironment{direta}{}{
\begin{flushright}
    Q.E.D
\end{flushright}
}
\newenvironment{contradicao}{A prova é por contradição.}{
\begin{flushright}
    Q.E.D
\end{flushright}
}
\newenvironment{contrapositiva}{A prova é pela contrapositiva.}{
\begin{flushright}
    Q.E.D
\end{flushright}
}
\newcommand{\definition}[3][x]{\{#1|#1 \in \mathbb{#2},#3\}}
\newcommand{\natura}{\mathbb{N}}
\newcommand{\integer}{\mathbb{Z}}
\newcommand{\real}{\mathbb{R}}
\newenvironment{pif}[3][1]{
Tomando $n = #1$ então $#2$ e $#3$

Agora vamos assumir que a expressão é valida para $#1 > n \leq k$ e vamos provar que funciona para $n=k+1$.
}{}
\newcommand{\arranjo}[2]{A_{#1,#2}=\frac{#1!}{(#1-#2)!}}
\newcommand{\arranjoform}[2]{\frac{#1!}{(#1-#2)!}}
\newcommand{\conbination}[2]{C_{#1,#2}=\frac{#1!}{#2!(#1-#2)!}}
\newcommand{\conbinationform}[2]{\frac{#1!}{#2!(#1-#2)!}}
\newcommand{\rconbination}[2]{CR_{#1,#2}=\frac{(#1+#2-1)!}{#2!(#1-1)!}}
\newcommand{\clinear}[1]{\foreach \i in {1,...,#1}{x_\i+}}
\newcommand{\fatorial}[2]{\foreach \i in {#1,...,#2}{\i.}}
\newcommand{\soma}[2]{\sum_{n=#1}^{\infty}#2}
\newcommand{\texto}{\color{mygray}\blindtext\color{black}}



%--------------%--------------%----------------------%-----------%--------------

\title{Projeto Wikiaves (WAV) x SpeciesLink (SLI)}
\author{Universidade Federal do ABC}
\date{Dezembro, 2020}

\begin{document}


\maketitle

%\textbf{Considerações}
\begin{enumerate}
\item 
\end{enumerate}
\newpage

\section{Resumo}

No atual cenário de escassez de dados quanto a distribuição de espécies de aves, a ciência-cidadã conquista um ofício relevante: a capacidade de ampliar a quantidade de dados vigente. O esforço de cientistas para acumular registros ornitológicos não é ineficaz, de fato, tais bancos de dados possuem uma vasta quantidade de dados. Entretanto, por meio de indivíduos cativados pela ornitologia constituem-se bancos de dados que reúnem uma quantidade superior de registros municipais. O presente estudo visa discutir a proximidade dos bancos de dados construídos por especialistas e os construídos por cientistas cidadãos do portal Wikiaves no estado de São Paulo.  

\section{Introdução}

Ecologia é o estudo da distribuição e abundância dos organismos \cite{begon2009}. Determinar os padrões espaciais e temporais da biodiversidade tem implicações diretas sobre o manejo de recursos e serviços naturais \cite{groom2006,Roque2018}, consideradas as diferentes escalas em que ela se manifesta – desde o nível molecular até o nível da paisagem, incluindo a riqueza de espécies (MAGURRAN et al., 2004). No período em que parte considerável da biota do planeta está ameaçada pela atividade humana, tal conhecimento pode reduzir efeitos que culminem em extinções prematuras \cite{groom2006}.

Porém, o número de pesquisadores aptos a conduzir a descrição pormenorizada dos referidos padrões é exígua ante a demanda \cite{Amano2016,Greenwood2007}. No Brasil, por exemplo, embora haja uma notável quantidade de aves que, no estado de São Paulo, são listadas desde o final do século XIX \cite{Silveira2011}, a ausência de uma quantidade abundante de registros de tais, prejudica o monitoramento da avifauna no estado. Esse cenário cria lacunas nas informações quanto a biodiversidade de tais espécies \cite{Amano2016} acarretando impasses quanto a conservação delas. 
	

Tornam-se, portanto, necessárias estratégias que otimizem tempo e esforço destinados a descrever os padrões apresentados \cite{Amano2016,Greenwood2007}. O recrutamento massivo de leigos para o cumprimento de alguma das etapas desse tipo de levantamento é uma delas \cite{Lepczyk2005,Phillips2014,Tredick2017,Horns2018}. Reconhece-se esse esforço como ciência cidadã \cite{Kullenberg2016} que, além de proporcionar informações que auxiliam o estudo da Ecologia, promove a premência da conservação e educação ambiental \cite{DiasdaSilva2019,Bonney2016}.


Esses cientistas-cidadãos atuam principalmente na coleta de informações em campo, aprimorando gradualmente sua performance \cite{Phillips2014,Kieslinger2019}, sob tutela de um cientista profissional (com formação acadêmica e vinculado a algum órgão de pesquisa). A qualidade e validade dos bancos de dados assim obtidos deve ser testada para a plena aplicação dos resultados advindos de sua interpretação \cite{Phillips2014,Tredick2017,Kieslinger2019}. Devido a alta quantidade de indivíduos envolvidos, estes bancos de dados são capazes de apresentar uma elevada quantidade de registros \cite{Alexandrino2018}.


Dadas as características marcantes e diagnósticas de grande parte das espécies de aves, sua determinação é plenamente possível por um iniciado com treino moderado. A disseminação global da atividade de birdwatching indica o apelo popular desse táxon \cite{Lepczyk2005,Alexandrino2018}. A ocorrência de populações de aves em determinada localidade é empregada amplamente como um indicador de condição ambiental \cite{Lepczyk2005,Greenwood2007,Schubert2019}, inclusive como um descritor alternativo e correlacionado à diversidade de outros grupos zoológicos e botânicos. Angariar a contribuição de indivíduos ordinários para o mapeamento seria, nesse caso, um modo de descrever o panorama do sistema ecológico usando essa mão-de-obra abundante e motivada \cite{Lepczyk2005,Klemann-Junior2017,Alexandrino2018}. 


Contudo, a ciência cidadã apresenta limitações. Para o monitoramento de aves, a primeira a se tratar é que os registros destas bases são dados de acordo com as coordenadas municipais e não são georreferenciados, o que pode ocasionar divergências quanto ao tratamento de dados, entretanto, não é por esse motivo que devem ser desconsiderados \cite{Neto2017}. Além disso, outra limitação que deve ser levada em consideração, antes de utilizar tais bases, é a incapacidade dos observadores de acessar terras privadas onde residem grande parte das espécies raras \cite{Lepczyk2005}. Dito isso, embora a ciência cidadã tenha uma vasta capacidade de reunir dados é preciso ter cautela quanto à utilização deles para projetos que exijam determinado rigor cientifico \cite{Kieslinger2019}.


Portanto, faz-se necessária a avaliação de se (e quanto) essa descrição baseada no trabalho dos cientistas-cidadãos desvia-se daquela obtida sem sua participação, valendo-se apenas dos esforços dos cientistas profissionais \cite{Klemann-Junior2017}. Esse estudo tem como finalidade discutir essas questões, baseado em duas bases de dados do estado de São Paulo: O sítio SpeciesLink (SLI) que reúne registros em coleções biológicas institucionais, adquiridas primordialmente durante a atividade de pesquisadores, esse tem sido um tipo de fonte fundamental para estudos envolvendo a distribuição da biodiversidade \cite{Horns2018}, e o sítio WikiAves (WAV) que reúne registros fotográficos e fonográficos de espécies em território brasileiro conduzidos por populares e com curadoria sob regência de especialistas (CUNHA, 2014; WikiAves 2020), até o dia 25/04/2020, contava com 3.119.856 registros de 33.918 contribuintes para 1.890 espécies.



\input{Sections/03}

\section{Metodologia}

A princípio, foram coletados, previamente, os dados provenientes dos cientistas cidadãos através do portal Wikiaves (WikiAves, 2020), além disso, também aqueles provenientes de cientistas pelas coleções: Carnegie Museum Birds Collection (CM-Birds), Coleção de Sons (CSUEL), Fonoteca Neotropical Jacques Vielliard (FNJV), Museum of Comparative Zoology (HU-Zoo), Coleção Zoológica de Referência da Seção de Vírus Transmitidos por Artrópodos - Banco de Aves (IAL-aves), Coleção de Aves (MCP-Aves), Museum of Vertebrate Zoology - Brazilian records (MVZ\_BR), Coleção de Aves do Museu de Zoologia (MZUEL-Aves), OBIS Brasil (OBIS\_BR), Sistema de Informação do Programa Biota/Fapesp (SinBiota), NMNH Extant Specimen and Observation Records (US-Animalia), Coleção de Aves do Museu de Zoologia da UNICAMP (ZUEC-AVE), Coleção Audiovisual do Museu de Zoologia "Adão José Cardoso" da UNICAMP - Coleção de vídeos (ZUEC-VID) disponíveis na rede speciesLink (http://www.splink.org.br). Com propósitos de equiparação dos bancos de dados, todas as espécies foram identificadas de acordo com o a Lista comentada das aves do Brasil pelo Comitê Brasileiro de Registros Ornitológicos (PIACENTINI, 2015). As informações foram coletadas no período de janeiro a fevereiro de 2020.

Para responder às questões supracitadas, serão utilizadas as seguintes variáveis oriundas dos dois bancos de dados: o número de registros de cada espécie por município, o número de espécies registradas por município e suas versões log-transformadas (base 10). O número de habitantes, a área de cada município e as coordenadas geográficas de suas sedes serão obtidas no sítio do IBGE (2020). Os biomas predominantes em cada município serão determinados de acordo com o sítio MapBiomas (2020).

A partir desses dados, análises univariadas descreverão a distribuição de cada descritor para os municípios valendo-se de estatísticas de posição (média, mediana), dispersão (amplitude, interquartis, desvio-padrão), que transmitirão a variabilidade dos dados, e distribuições de frequências, que será visualizada por meio de histogramas (GOTTELI & ELLISON et al., 2016, BORCARD et al., 2018). Será utilizada a ferramenta Microsoft Excel® (2019) para realizar as análises descritas acima.
Análises bivariadas descreverão as relações das variáveis pareadas, por meio de correlação e regressão, em versões paramétricas ou não (GOTTELI & ELLISON et al., 2016, BORCARD et al., 2018). Comparações entre curvas serão feitas por análise de covariância (GOTTELI & ELLISON et al., 2016, BORCARD et al. 2018). Elas serão realizadas por meio dos programas Microsoft Excel® e Past (HAMMER, Ø, 2019).

Análises multivariadas descreverão as relações de similaridade entre municípios quanto à constituição específica e quanto aos valores dos fatores explanatórios (GOTTELI & ELLISON et al., 2016, BORCARD et al., 2018). Para esta dimensão de analise utilizar-se-á, por meio do programa Past, o índice de Jaccard (BORCARD et al., 2018), que abrange a dissimilaridade entre duas amostras baseando-se na presença de espécies em certa região, a partir das matrizes fornecidas por tal será possível compara-las por meio do teste de Mantel (BORCARD et al., 2018). Este método de análise, também, permitirá que sejam comparadas a distância euclidiana por município, que será obtida a partir dos bancos de dados, com a distância geográfica já conhecida entre eles. Mapas temáticos apresentarão as variáveis em sua distribuição geográficas. As abordagens discriminadas serão conduzidas com o programa R (R CORE TEAM, 2013).


\subsection{Materiais e Métodos}

\subsection{Etapas da pesquisa}

\section{Mapas}

\hrulefill


\hrulefill

\begin{resposta}
Os mapas facilitam a observação espacial dos fatores abordados na seção 2. Ademais, possibilitam a observação de como se distribuem os registros ou espécies em cada banco de dados pelo estado de São Paulo. A cada subseção serão mostrados mapas de acordo com os devidos temas.
\end{resposta}

\subsection{Altitude}

\begin{figure}[h!]
\centering
\includegraphics[height = 5cm]{Imagens/M01.png}
\\{\scriptsize Figura 22. Distribuição espacial dos municípios do estado de São Paulo em classes segundo a altitude (m) de sua sede.}
\end{figure}

\begin{resposta}
A distribuição de altitude no estado se dá de maneira regular. Ainda, a maior parte dos municípios se apresenta em uma faixa específica dentre os valores determinados.
\end{resposta}

\subsection {Área}

\begin{figure}[h!]
\centering
\includegraphics[height = 5cm]{Imagens/M02.png}
\\{\scriptsize Figura 23. Distribuição espacial dos municípios do estado de São Paulo em classes segundo a área (Log10 km2).}
\end{figure}

\begin{resposta}
A distribuição do $\log_{10}$ da área, também é regular. Há uma quantidade limitada de municípios com áreas muito baixas, ou seja, os outliers.
\end{resposta}


\subsection {População}

\newpage

\begin{figure}[h!]
\centering
\includegraphics[height = 5cm]{Imagens/M03.png}
\\{\scriptsize Figura 24. Distribuição espacial dos municípios do estado de São Paulo em classes segundo o tamanho da população humana (Log10 indivíduos).}
\end{figure}

\begin{resposta}
A distribuição do $\log_{10}$ da população também é regular, a cidade de São Paulo é um outlier evidente no mapa.
\end{resposta}

\subsection {Registros}

\begin{figure}[h!]
\centering
\includegraphics[width=17cm]{Imagens/M04.png}
\\{\scriptsize Figura 25. Distribuição espacial dos municípios do estado de São Paulo em classes segundo o número de registros (Log10, esquerda) e o número de espécies (Log10, direita) em cada banco de dados (WAV = superior, SLI = central, WAV2 = inferior).}
\end{figure}

\begin{resposta}
Há uma desigualdade perceptível entre os bancos de dados quanto a distribuição espacial da quantidade de registros. O mapa do Wikiaves é mais saturado que o do SpeciesLink, a quantidade de cidades sem registro, também, torna-se ainda mais evidente: enquanto o Wikiaves cobre quase todo o estado, o SpeciesLink não cobre metade disto. 

Não obstante, há similaridades observadas nos mapas, eles mostram uma expressão mais forte no litoral e, conforme se movimenta para  o interior há uma diminuição na quantidade de registros, enfatizando a relevância dos parâmetros de latitude e longitude.
\end{resposta}

\subsection {Espécies}

\begin{figure}[h!]
\centering
\includegraphics[width=17cm]{Imagens/M05.png}
\end{figure}

\begin{resposta}
As mesmas divergências e semelhanças mencionadas na subseção anterior são encontradas para a quantidade de espécies, de forma menos chamativa. 
\end{resposta}

\section{Conclusões e perspectivas de trabalhos futuros}

\input{Sections/07}

%\bibliography{REF}


\end{document}
